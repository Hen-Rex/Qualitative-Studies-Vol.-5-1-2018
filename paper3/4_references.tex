\label{paper3:references}
\begin{thebibliography}{99}
%%%%%%%%%%%%%%%%%%%%%%%%%%%%%%%%%%%%%%%%%%%%%%%%%%%%%%%%
%%%%%%%%%%%%%%% START 9_references.tex %%%%%%%%%%%%%%%%%
%%%%%%%%%%%%%%%%%%%%%%%%%%%%%%%%%%%%%%%%%%%%%%%%%%%%%%%%

\item Alvesson, M. \& Kärreman, D. (2011). Qualitative Research and Theory Development. Mystery as Method. London: Sage.
\item Brinkmann, S. (2014). Doing without data. Qualitative Inquiry, vol. 20(6), pp. 720-725. doi: 10.1177/1077800414530254
\item Davies, B. (2011). Listening: a radical pedagogy. Challenging gender: Normalization and beyond, 1-17. Working paper: \url{https://www.google.co.uk/url?sa=t\&rct=j\&q=\&esrc=s\&source=web\&cd=3\&cad=rja\&uact=8\&ved=0ahUKEwja0Jb-26jaAhWKJsAKHXZpC-AQFgg2MAI\&url=https\%3A\%2F\%2Fis.muni.cz\%2Fel\%2F1423\%2Fpodzim2009\%2Fsoc932\%2Fum\%2F8834758\%2FListening-_a_radical_pedagogy_11_08_09.pdf\&usg=AOvVaw1h_rNGmHcQ_iK9eROTrfTU}.
\item Davies, B. (2016). Emergent Listening. Qualitative Inquiry Through a Critical Lens, 73.
\item Daza, S., \& Gershon, W. S. (2015). Beyond Ocular Inquiry: Sound, Silence, and Sonification. Qualitative Inquiry, 1077800414566692.
\item Elias, N. (1956). Problems of involvement and detachment. The British Journal of Sociology, 7(3), 226-252.
\item Fachin, F. F. \& Langley, A. (2017). Researching Organizational Concepts Processually: The Case of Identity. In Cassell C., Cunliffe A. \& Grandy G. (Eds.) The Sage Handbook of Qualitative Business and Management Research Methods: History and Traditions, 308. Thousand Oaks, CA: Sage.
\item Feldman, D. C. (2012). The impact of socializing newcomers on insiders. In Wanberg, C. (ed.), The Oxford handbook of organizational socialization, 215-229.
\item Flaherty, M., \& Fine, G. A. (2001). Present, past, and future: Conjugating George Herbert Mead’s perspective on time. Time \& Society, 10, 147-161.
\item Gallagher, E. B., \& Sias, P. M. (2009). The new employee as a source of uncertainty: Veteran employee information seeking about new hires. Western Journal of Communication, 73(1), 23-46.
\item Helin, J., Hernes, T., Hjorth, D., \& Holt, R. (2014). Process is how process does. In J. Helin, T. Hernes, D. Hjorth, \& R. Holt (Eds), The Oxford Handbook of Process Philosophy and Organization Studies (pp. 1-16). Oxford, UK: Oxford University Press.
\item Jackson, A. Y., \& Mazzei, L. A. (2013). Plugging One Text into Another: Thinking with Theory in Qualitative Research. Qualitative Inquiry, 19(4), 261-271.
\item Jackson, A. Y., \& Mazzei, L. A. (2017). Thinking with theory: A new analytic for qualitative inquiry. In N. K. Denzin \& Y. S. Lincoln (Eds.), The SAGE handbook of qualitative research, 5th ed., (pp. 717-737). Thousand Oaks, CA: SAGE.
\item Langley, A., \& Tsoukas, H. (2017). Introduction: Process thinking, process theorizing and process researching. The SAGE Handbook of Process Organizational Studies, 1-25.
\item Mead, G. H. (1932). The Philosophy of the Present (this edition 2002). Amherst, New York: Prometheus Books. (Originally published: Chicago: Open Court Pub).
\item Mead, G. H. (1934). Mind, self, and society: From the standpoint of a social behaviorist (C. W. Morris, Ed.). Chicago, IL: The University of Chicago Press.
\item Nancy, J. L., \& Mandell, C. (2007). Listening. Fordham Univ Press.
\item Pierce, C.S. (1978). Pragmatism and abduction. In C. Hartshorne \& P. Weiss (Eds), Collected Papers vol. V (pp. 180-212). Cambridge, MA: Harvard University Press.
\item Revsbæk, L. (2014). Adjusting to the Emergent. A process theory perspective on organizational socialization and newcomer innovation. (Doctoral thesis). Aalborg, DK: Aalborg University Press.
\item Revsbaek, L. \& Tanggaard, L. (2015). Analyzing in the Present. Qualitative Inquiry, vol. 21(4), 376-387.
\item Shotter, J. (2010). Adopting a process orientation… in practice: Chiasmic relations, language, and embodiment in a living world. In Hernes, T. \& Maitlis, S. (Eds.), Process, sensemaking, and organizing (Vol. 1), Oxford, UK: Oxford University Press, pp. 70-101.
\item Shotter, J. (2015). On “relational things”: A new realm of inquiry—pre-understandings and performative understandings of people’s meanings. The emergence of novelty in organizations, 56-79.
\item Simpson, B. (2014). George Herbert Mead (1863-1931). In J. Helin, T. Hernes, D. Hjorth, \& R. Holt (Eds.), The Oxford Handbook of Process Philosophy and Organization Studies (pp. 272-286). Oxford, UK: Oxford University Press.
\item St. Pierre, E. A. (1997). Methodology in the fold and the irruption of transgressive data. International Journal of Qualitative Studies in Education, 10(2), 175-189.
\item St. Pierre, E.A. (2011). Post qualitative research: The critique and the coming after. In N.K. Denzin \& Y.S. Lincoln (Eds), The SAGE Handbook of Qualitative Research, 4th ed. (pp. 611-625). Thousand Oaks, CA: Sage. 
\item Stacey, R.D. (2010). Complexity and Organizational Reality. Uncertainty and the Need to Rethink Management after the Collapse of Investment Capitalism, 2nd Ed. Oxon, UK: Routledge. 
\item Stacey, R. (2012). Tools and Techniques of Leadership and Management. Meeting the Challenge of Complexity. Oxon, UK: Routledge.

%##################################################################################################################################################################################################################################
\end{thebibliography}