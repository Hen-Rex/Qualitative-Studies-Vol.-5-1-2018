    \begin{adjmulticols}{1}{10mm}{10mm}
\label{paper3:abstract}
    \bigskip
    \begin{otherlanguage}{english}
    {\small
    \fadebreak
%%%%%%%%%%%%%%%%%%%%%%%%%%%%%%%%%%%%%%%%%%%%%%%%%%%%%%%%
%%%%%%%%%%%%%%%% START ABSTRACT.tex %%%%%%%%%%%%%%%%%%%%
%%%%%%%%%%%%%%%%%%%%%%%%%%%%%%%%%%%%%%%%%%%%%%%%%%%%%%%%

\noindent Theory, and the traditions of thought available and known to us, give shape to what we are able to notice of our field of inquiry, and so also of our practice of research. Building on G. H. Mead’s Philosophy of the Present (1932), this paper draws attention to ‘emergent events’ of analysis when working abductively with interview data in a process of re-experiencing interview material through listening to audio recordings of qualitative research interviews. The paper presents an emergent event of analysis in which the theoretical argument of (the researcher’s) Self as a process of becoming in responsive relating to (case study) others is made generative as a dynamic in and of case study analysis. Using a case of being a newcomer (to research communities) researching newcomer innovation (of others), ‘resonant experience’ is illustrated as a heuristic in interview analysis to simultaneously deconstruct/reconstruct dichotomous concept categories known to organize the research literature in a field. 
%
\medskip
%%%%% KEYWORDS
%
\\\noindent{\scshape keywords}\hspace*{1.75em}{analysis, interview analysis, process ontology, resonant experience, emergent event, G. H. Mead}.





    } % ends font family

    \fadebreak

    \end{otherlanguage}

    \end{adjmulticols}