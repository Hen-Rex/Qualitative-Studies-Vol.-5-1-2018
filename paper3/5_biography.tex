    \fadebreak{}
\subsection{About the Author}
%    \marginnote{please write a mini-biography/description of yourself, qualifications, and current work, and anything else you might think of :)}
\label{paper3:colophon}
%%%%%%%%%%%%%%%%%%%%%%%%%%%%%%%%%%%%%%%%%%%%%%%%%%%%%%%%
%%%%%%%%%%%%% START ACKNOWLEDGEMENTS.tex %%%%%%%%%%%%%%%
%%%%%%%%%%%%%%%%%%%%%%%%%%%%%%%%%%%%%%%%%%%%%%%%%%%%%%%%

\textbf{Line Revsbæk} PhD, MSc in psychology, is an associate professor at \textit{the Department of Learning and Philosophy}, Aalborg University, Aalborg, Denmark. She has held positions at the University of Southern Denmark and was a visiting student at the Complexity and Management Group, Hertfordshire University, UK, during her doctoral studies. Her research focuses on organizational socialization, employee induction and the social dynamics of collaboration and organizational life. She works from process ontology, pragmatism and complexity theory perspectives to develop research practice and methodology in terms of complex responsive processes. She has been engaged in participatory action research on employee induction/onboarding in academic, state government and private sector organizations, and is currently (together with Lærke Gelineck Berg and Søren Willert) researching collaborative reflexive writing as a method of leadership development in a Danish municipality. Recent publications include Revsbæk, L. (2014), Adjusting to the Emergent. A process theory perspective on organizational socialization and newcomer innovation, Aalborg University Press, Aalborg; Revsbæk, L. \& Tanggaard, L. (2015), Analyzing in the Present, Qualitative Inquiry, vol. 21(4), 376-387; Mosleh, W. S. \& Revsbæk, L. (2017), Fieldworking relational complexity: Entangled in managerial dynamics, conference paper, EGOS, July 6th-8th, Copenhagen; Ylirisko, S., Revsbæk, L. \& Buur, J., Resourcing experience in co-design, Conference proceedings from DTRS11 (2017); Revsbæk, L. (2016), Making methodology a matter of process ontology, Organisation und Methode. Beiträge der Kommision Organisationspädagogik, Springer VS (eds. Göhlich, M., Weber, S. M., Schröer, A. and Schemmann, M., 2016).      