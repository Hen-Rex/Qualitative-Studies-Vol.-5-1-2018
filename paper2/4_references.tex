\label{paper2:references}
\begin{thebibliography}{99}
%%%%%%%%%%%%%%%%%%%%%%%%%%%%%%%%%%%%%%%%%%%%%%%%%%%%%%%%
%%%%%%%%%%%%%%% START 9_references.tex %%%%%%%%%%%%%%%%%
%%%%%%%%%%%%%%%%%%%%%%%%%%%%%%%%%%%%%%%%%%%%%%%%%%%%%%%%

\item Ammon, G. (Ed.) (1974). \textit{Gruppendynamik der kreativität}. München: Kindler TB.
\item Baarts, C. (2015). Autoetnografi.  In S. Brinkmann \& L. Tanggaard (Eds.) \textit{Kvalitative metoder: En grundbog} (2nd ed) (p. 169–167). Copenhagen: Hans Reitzels Forlag. 
\item Barcellos, L. (2012). Music, meaning, and music therapy under the light of the Molino/Nattiez Tripartite Model. \textit{Voices: A World Forum for Music Therapy}, 12(3).
\item Bergstrøm-Nielsen, C. (2010). Graphic notation –-- the simple sketch and beyond. \textit{Nordic Journal of Music Therap}y, 19(2), 162–177
\item Brinkmann, S. (2014). \textit{Stå fast}. Copenhagen: Gyldendal
\item Casement, P. (2014). \textit{On learning from the patient}. London and New York: Routledge. (first published by Tavistock 1985).
\item Coomans, A. (2016). \textit{Moments of resonance in musical improvisation with persons with severe dementia}. Dissertation, Department of Communication and Psychology, Aalborg University. 
\item Eschen, J. T. (2002). \textit{Analytical music therapy}. London: Jessica Kingsly Publishers.
\item Haneishi, E. (2001). Effects of a music therapy voice protocol on speech intelligibility, vocal acoustic measures, and mood of individuals with Parkinson's disease. \textit{Journal of Music Therapy}, 38(4), 273–290
\item Hannibal, N. (2014). Implicit and explicit mentalization in music therapy in the psychiatric treatment of people with borderline personality disorder. In J. De Backer \& J. Sutton (Eds.), \textit{The music in music therapy: Psychodynamic music therapy in Europe: Clinical, theoretical and research approaches} (p. 211- 223). London: Jessica Kingsley Publishers.
\item Hart, S. (2006). \textit{Hjerne, samhørighed, personlighed. Introduktion til neuroaffektiv udvikling}. Copenhagen: Hans Reitzels Forlag.
\item Hart, S., \& Bentzen, M. (2013). \textit{Jagten på de nonspecifikke faktorer i psykoterapi med børn}. Copenhagen: Hans Reitzels Forlag.  
\item Jacobsen, B., Tanggaard, L. \& Brinkmann, S. (2015). Fænomenologi. In S. Brinkmann \& L. Tanggaard (Eds.), \textit{Kvalitative metoder}. En grundbog (2nd ed.) (p. 217–239). Copenhagen: Hans Reitzels Forlag. 
\item Jenny, H. (2004). \textit{Cymatics: A study of wave phenomena and vibrations}. Newmarket NH: MACROmedia.
\item Lindvang, C. (2010). \textit{A field of resonant learning: Self-experiential training and the development of music therapeutic competencies}. PhD dissertation, Department of Communication and Psychology, Aalborg University.
\item Lindvang, C. (2015). Kompleksitet i læreprocesser og terapi. In T. Hansen (Ed.), \textit{Det ubevidstes potentiale: Kybernetisk psykologi i anvendelse} (p. 100–121). Copenhagen: Frydenlund.
\item Malloch, S., \& Trevarthen, C. (Eds.) (2010). \textit{Communicative musicality: Exploring the basis of human companionship}. Oxford: Oxford University Press.
\item Meier, N., \& Wegener, C. (2016). Writing with resonance. \textit{Journal of Management Inquiry}, 26(2), 193–201. 
\item Pedersen, I. N. (1997). The music therapist's listening perspectives as source of information in improvised musical duets with grown-up, psychiatric patients, suffering from schizophrenia. \textit{Nordic journal of Music Therapy}, 6(2), 98–112.
\item Pedersen, I. N. (2007). \textit{Counter transference in music therapy. A phenomenological study on counter transference used as a clinical concept by music therapists working with musical improvisation in adult psychiatry}. Dissertation, Department of Communication and Psychology, Aalborg University.
\item Pedersen, I. N. (2007a). Musikterapeutens disciplinerede subjektivitet. \textit{Psyke \& Logos. Tema: Musik og Psykologi}, 28(1), 358–384. 
\item Priestley, M. (1994). \textit{Essays on analytical music therapy}. Phoenixville, PA: Barcelona Publishers.
\item Ridder, H. M. (2017). Selvregulering og dyadisk regulering i musikterapi med demensramte. In C. Lindvang \& B. D. Beck (Eds.), \textit{Musik, krop og følelser. Neuroaffektive processer i musikterapi} (p. 197-210). Copenhagen: Frydenlund Academics.
\item Ridder, H. M. \& Bonde, L. O. (2014). Musikterapeutisk forskning: Et overblik. In L. O. Bonde (Ed.), \textit{Musikterapi. Teori, uddannelse, praksis, forskning: En håndbog om musikterapi i Danmark} (p. 407–425). Aarhus: Forlaget Klim.
\item Robarts, J. Z. (2000). Music therapy and adolescents with anorexia nervosa. \textit{Nordic Journal of Music Therapy}, 9(1), 3–12.
\item Robbins, A. (1998). \textit{Between therapists. The processing of transference/counter transference material}. New York: Human Science Press.
\item Robson, C. (2011). \textit{Real world research} (3rd edition). London: Wiley.
\item Rosa, H. (2016). \textit{Resonanz: Eine Soziologie der Weltbeziehung}. Berlin: Suhrkamp.
\item Rycroft, C. (1972). \textit{A critical dictionary of psychoanalysis}. Harmondsworth: Penguin Books (First published by Nelson, 1968)
\item Skov, V. (2013). \textit{Art therapy. Prevention against the development of depression}. Dissertation, Department of Communication and Psychology, Aalborg University.
\item Smeijsters, H. (2012). Analogy and metaphor in music therapy. Theory and practice. \textit{Nordic Journal of Music Therapy}, 21(3), 227-249.
\item Stern, D. N. (1985/1991). \textit{Barnets interpersonelle univers}. Copenhagen: Hans Reitzels Forlag.
\item Stern, D. (2004). \textit{The present moment in psychotherapy and everyday life}. New York: W.W. Norton \& Company, Inc.
\item Stern, D. (2010). \textit{Forms of vitality}. Oxford: Oxford University Press.
\item Storm, S. (2013). \textit{Research into the development of voice assessment in music therapy}. Dissertation, Department of Communication and Psychology, Aalborg University.
\item Watzlawick, P. (2004). \textit{When music resonates with emotions: Meanings and significance in the narrative of young music therapy students}. RevistaCientífica da FAP, 1-15.
\item Wigram, T. (2004). \textit{Improvisation: Methods and techniques for music therapy clinicians, educators and students}. London: Jessica Kingsley Publishers.
\item Zaki, J., \& Ochsner, K. N. (2012). The neuroscience of empathy: Progress, pitfalls and promise. \textit{Nature Neuroscience}, 15(5), 675–80.
\item Zitzewitz, P. W. (2011). \textit{The handy physics answer book}. Canton Township, Michigan: Visible Ink Press.

%##################################################################################################################################################################################################################################
\end{thebibliography}