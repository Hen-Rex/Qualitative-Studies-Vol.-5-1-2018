    \begin{adjmulticols}{1}{10mm}{10mm}
\label{paper2:abstract}
    \bigskip
    \begin{otherlanguage}{english}
    {\small
    \fadebreak
%%%%%%%%%%%%%%%%%%%%%%%%%%%%%%%%%%%%%%%%%%%%%%%%%%%%%%%%
%%%%%%%%%%%%%%%% START ABSTRACT.tex %%%%%%%%%%%%%%%%%%%%
%%%%%%%%%%%%%%%%%%%%%%%%%%%%%%%%%%%%%%%%%%%%%%%%%%%%%%%%

\noindent Resonance is often used to characterize relationships, but it is a complex concept that explains quite different physical, physiological and psychological processes. With the aim of gaining deeper insight into the concept of resonance, a group of ten music therapy researchers, all colleagues, embarked on a joint journey of exploration. This included an aim of letting the internal learning process be disseminated in a way that could give others insight, not only from the findings, but also from the process. Findings include a dual understanding of resonance as (1) a visible and ordered phenomenon consisting of physical vibrations and acoustic sounding that offers a clear logic, and (2) a metaphorical conceptualization used to describe and understand complex psychological processes of human relationships. The process of collaborative writing led to the discovery or development of a nine-step procedure including different collaborative resonant writing procedures and musical improvisation, as well as of a series of metaphors to explain therapeutic interaction, resonant learning and ways of resonant exploration. 
%
\medskip
%%%%% KEYWORDS
%
\\\noindent{\scshape keywords}\hspace*{1.75em}{resonance, collaborative thinking, musicking and writing, improvisation, music therapy, qualitative research}.








    } % ends font family

    \fadebreak

    \end{otherlanguage}

    \end{adjmulticols}