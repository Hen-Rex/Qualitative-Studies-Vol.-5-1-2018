    \begin{adjmulticols}{1}{10mm}{10mm}
\label{paper4:abstract}
    \bigskip
    \begin{otherlanguage}{english}
    {\small
    \fadebreak
%%%%%%%%%%%%%%%%%%%%%%%%%%%%%%%%%%%%%%%%%%%%%%%%%%%%%%%%
%%%%%%%%%%%%%%%% START ABSTRACT.tex %%%%%%%%%%%%%%%%%%%%
%%%%%%%%%%%%%%%%%%%%%%%%%%%%%%%%%%%%%%%%%%%%%%%%%%%%%%%%

\noindent Given that qualitative researchers have (rightly) abandoned the idea of social scientific truths as mirrors of nature, what kind of truth do we hope to provide to our readers? In other words, what is the point of reading qualitative research? Taking inspiration from Paul Ricoeur’s distinction between a hermeneutics of suspicion and a hermeneutics of faith, this article sketches out two possible answers. It first presents a \textit{critical approach} that exposes hidden truths to educate and emancipate its readers. The concept of ‘critique’ has recently come under scrutiny, however, with postcritical scholars denouncing its tautological reasoning, its reductionist analytical strategies and its arrogant approach to other people. Acknowledging these criticisms, the article then goes on to present a \textit{phenomenological approach} that points out unnoticed truths to reverberate and resonate with its readers. It is argued that this self-consciously ‘weak’ approach helps us circumvent the analytical issues currently associated with critique.
%
\medskip
%%%%% KEYWORDS
%
\\\noindent{\scshape keywords}\hspace*{1.75em}{critique, interpretation, phenomenology, post-critique, resonance}.








    } % ends font family

    \fadebreak

    \end{otherlanguage}

    \end{adjmulticols}