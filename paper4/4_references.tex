\label{paper4:references}
\begin{thebibliography}{99}
%%%%%%%%%%%%%%%%%%%%%%%%%%%%%%%%%%%%%%%%%%%%%%%%%%%%%%%%
%%%%%%%%%%%%%%% START 9_references.tex %%%%%%%%%%%%%%%%%
%%%%%%%%%%%%%%%%%%%%%%%%%%%%%%%%%%%%%%%%%%%%%%%%%%%%%%%%

\item Aagaard, J. (2017). Introducing Postphenomenological Research: A Brief and Selective Sketch of Phenomenological Research Methods. \textit{International Journal of Qualitative Studies in Education, vol. 30(6)}, p. 519-533.
\item Anker, E. \& Felski, R. (2017). \textit{Critique and Postcritique}. Durham: Duke University Press.
\item Barnwell, A. (2016) Entanglements of Evidence in the Turn Against Critique. \textit{Cultural Studies, vol. 30(6)}, p. 906-925.
\item Becker, H. (2003). The Politics of Presentation: Goffman and Total Institutions. \textit{Symbolic Interaction, vol. 26(4)}, p. 659-669.
\item Borgmann, A. (1984). \textit{Technology and the Character of Contemporary Life}. Chicago: Chicago University Press.
\item Bourdieu, P. (1983). Erving Goffman, Discoverer of the Infinitely Small. \textit{Theory Culture \& Society, vol. 2(1)}, p. 112-113.
\item Brinkmann, S. (2012). \textit{Qualitative Inquiry in Everyday Life: Working with Everyday Life Materials}. London: Sage.
\item Carman, T. (2006). The Principle of Phenomenology. In: C. Guignon (ed.). \textit{The Cambridge Companion to Heidegger}. Cambridge: Cambridge University Press.
\item Colaizzi, P. (1978). Learning and Existence. In: R. Valle \& M. King (eds.). Existential-Phenomenological Alternatives for Psychology. New York: Oxford University Press.
\item Denzin, N., Lincoln, Y. \& Giardina, M. (2006). Disciplining Qualitative Research. \textit{International Journal of Qualitative Studies in Education, vol. 19(6)}, p. 769-782.
\item Dreyfus, H. (1980). Holism and Hermeneutics. \textit{Review of Metaphysics, vol. 34(1)}, p. 3-23.
\item Dreyfus, H. (1991). \textit{Being-in-the-World: A Commentary on Heidegger’s Being and Time, Division I}. Cambridge: MIT Press.
\item Felski, R. (2008). \textit{Uses of Literature}. Malden, MA: Blackwell Publishing.
\item Felski, R. (2015). \textit{The Limits of Critique}. Chicago: The University of Chicago Press.
\item Ferguson, M. (2009). Resonance and Dissonance: The Role of Personal Experience in Iris Marion Young’s Feminist Phenomenology. In: A. Ferguson \& M. Nagel (eds.). \textit{Dancing with Iris: The Philosophy of Iris Marion Young}. Oxford: Oxford University Press.
\item Fleissner, J. (2017). Romancing the Real: Bruno Latour, Ian McEwan, and Postcritical Monism. In: E. Anker \& R. Felski (eds.). \textit{Critique and Postcritique}. Durham: Duke University Press.
\item Frieden, K. (1990). \textit{Freud’s Dream of Interpretation}. Albany, NY: State University of New York Press.
\item Gergen, K. (1973). Social Psychology as History. \textit{Journal of Personality and Social Psychology, vol. 26(2)}, p. 309–20.
\item Goffman, E. (1963). \textit{Behavior in Public Places: Notes on the Social Organization of Gatherings}. New York: The Free Press.
\item Greiffenhagen, C. \& Sharrock, W. (2008). Where do the Limits of Experience Lie? Abandoning the Dualism of Objectivity and Subjectivity. \textit{History of the Human Sciences, vol. 21(3)}, p. 70-93.
\item Heidegger, M. (2008). \textit{Being and Time}. New York: Harper Perennial.
\item Ihde, D. (1998). \textit{Expanding Hermeneutics: Visualism in Science}. Evanston, IL: Northwestern University Press.
\item Jameson, F. (2013). \textit{The Political Unconscious: Narrative as a Socially Symbolic} \textit{Act}. London: Routledge Classics.
\item Josselson, R. (2004). The Hermeneutics of Faith and the Hermeneutics of Suspicion. \textit{Narrative Inquiry, vol. 14(1)}, p. 1-28.
\item Latour, B. (1996). \textit{Aramis, or the Love of Technology}. Cambridge: Harvard University Press.
\item Latour, B. (2004a). How to Talk About the Body? The Normative Dimension of Science Studies. \textit{Body and Society, vol. 10(2-3)}, p. 205–229.
\item Latour, B. (2004b). Why Has Critique Run out of Steam? From Matters of Fact to Matters of Concern. \textit{Critical Inquiry, vol. 30(2)}, p. 225-248.
\item Latour, B. (2005). \textit{Reassembling the Social: An Introduction to Actor-Network Theory}. New York: Oxford University Press.
\item Mann, B. (2009). Iris Marion Young: Between Phenomenology and Structural Injustice. In: A. Ferguson \& M. Nagel (eds.). \textit{Dancing with Iris: The Philosophy of Iris Marion Young}. Oxford: Oxford University Press.
\item Marx, K. (1971). \textit{Capital (Vol. III)}. London: Lawrence \& Wishart.
\item Moi, T. (2017). \textit{Revolution of the Ordinary: Literary Studies After Wittgenstein, Austin, and Cavell}. Chicago: The University of Chicago Press.
\item Paley, J. (2017). \textit{Phenomenology as Qualitative Research: A Critical Analysis of Meaning Attribution}. London: Routledge.
\item Parker, I. (1996). Discursive Complexes in Material Culture. In: J. Haworth (ed.). \textit{Psychological Research: Innovative Methods and Strategies}. London: Routledge.
\item Ricoeur, P. (1970). \textit{Freud and Philosophy: An Essay on Interpretation}. New Haven: Yale University Press.
\item Ricoeur, P. (1974). Existence and Hermeneutics. In: \textit{The Conflict of Interpretations: Essays in Hermeneutics}. Evanston, IL: Northwestern University Press.
\item Ricoeur, P. (1981). Science and Ideology. In: \textit{Hermeneutics and the Human Sciences}. Cambridge: Cambridge University Press. 
\item Rorty, R. (1979). \textit{Philosophy and the Mirror of Nature}. Princeton, NJ: Princeton University Press.
\item Scott-Baumann, A. (2009). \textit{Ricoeur and the Hermeneutics of Suspicion}. New York: Continuum.
\item Sedgwick, E. (2003). Paranoid Reading and Reparative Reading, or, You’re So Paranoid, You Probably Think This Essay is About You. In: \textit{Touching Feeling: Affect, Pedagogy, Performativity}. Durham: Duke University Press.
\item Sontag, S. (1990). Against Interpretation. In: \textit{Against Interpretation and Other Essays}. New York: Anchor Books.
\item Spivak, G. (1996). Bonding in Difference: Interview with Alfred Arteaga. In: D. Landry \& G. MacLean (eds.). \textit{The Spivak Reader: Selected Works of Gayatri Chakravorty Spivak}. New York: Routledge.
\item Thomson, I. (2009). Phenomenology and Technology. In: V. Hendricks, J. K. B. Friis, \& S. Pedersen (eds.). \textit{A Companion to the Philosophy of Technology}. Hoboken: Wiley-Blackwell.
\item Van Manen, M. (1990). \textit{Researching Lived Experience: Human Science for an Action Sensitive Pedagogy}. New York: SUNY Series in the Philosophy of Education.
\item Wittgenstein, L. (2009). \textit{Philosophical Investigations}. Oxford: Wiley-Blackwell.
\item Young, I.M. (1980). Throwing Like a Girl: A Phenomenology of Feminine Body Comportment, Motility and Spatiality. \textit{Human Studies, vol. 3(2)}, p. 137-156.

%##################################################################################################################################################################################################################################
\end{thebibliography}