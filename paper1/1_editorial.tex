%%%%%%%%%%%%%%%%%%%%%%%%%%%%%%%%%%%%%%%%%%%%%%%%%%%%%%%%
%%%%%%%%%%%%%%%%% START 1_ARTICLE.tex %%%%%%%%%%%%%%%%%%
%%%%%%%%%%%%%%%%%%%%%%%%%%%%%%%%%%%%%%%%%%%%%%%%%%%%%%%%

\lettrine[lines=2]{\bfseries\color{black}Q}{ualitative Studies} is back after a period of hibernation. We are excited and proud to bring this issue into the buzzing world of incidents, accidents, intentions, dissonances and dreams come true. We have relaunched the journal in a rhythm of two thematic issues per year. This first one is about resonance. In our editorial group, we have been preoccupied with the concept of resonance for a long time, and the process of producing this issue has been a welcome opportunity to dig deeper. \textit{Resonance} does in fact reverberate quite intuitively for many people, and we believe that most researchers share the ambition to do research and write about it in ways that evoke resonance--—in the research community, in the individual reader, in the field of practice from which the research springs. Resonance is a wonderful metaphor for being affected as a human being, while it is also a concept referring to a physical and tangible occurrence. Ingold (1990: 199) describes resonance as ‘rhythmic harmonisation of mutual attention’, which he argues to be ‘the very foundation of sociality’ (ibid: 196; see also Wikan, 1992). The German sociologist Hartmut Rosa goes even further and uses the concept to suggest an alternative to what he considers as an accelerated modern society’s alienating impact on our lives and to explore the relationship we have---or could have--–with the world around us (Rosa, 2016). In continuation of this, resonance needs room to occur and a sounding board. If our academic schedules and minds are packed to capacity, it leaves limited room for resonance to occur. If we judge our academic raison d’être by quantifiable impact indicators, we have a measly sounding board. Attention does not necessarily take a lot of time, but a hurried life is also often a non-attentional life. 
\par
In music theory, resonance explains the quality of a sound or the intensification and enrichment of a musical tone. Resonance is the ability to evoke a response, when a vibration produced in one object causes a vibration in another object. As researchers, we can seek to be resonant in our relationship with the object of inquiry, and we can strive to produce analyses and writing that resonate with our readers, writing that evokes understanding of, or relates to, lived experience. According to Latour (2004: 210), ‘there is nothing especially interesting, deep, profound, worthwhile in a subject \enquote{by itself}, […] ---a subject only becomes interesting, deep, profound, worthwhile when it resonates with others, is effected, moved, put into motion by new entities whose differences are registered in new and unexpected ways’. Talking about the layered, mediated experiences of differences, Latour argues that bodies are our common destiny and that having a body means learning to be affected: ‘the more you learn, the more differences exist’ (Latour, 2004: 213). In line with this, resonance makes it possible to grasp and talk about the ways in which we are affected by the world and the way we can affect the world. Resonance, then, goes beyond the words; it refers to potential relationships between writer, text and reader, and it is tied to our bodies, imagination, memory and experience (Meier \& Wegener, 2017). Being resonant and aiming for resonance is an ongoing quest in a noisy world on the move, and resonance is not achieved by screaming louder. Asserting what we believe to be true may require a raised voice, be it in critical research papers, blogging or media performances. It certainly requires ongoing dialogue with our students about the meaning of (their) life and education. A hurried, output-guided academia low on resonance is an academia at risk of depletion and shallowness. This is not only disastrous for the individual researcher and student who might wonder where to find joy and connectedness; it is also devastating to the shared and continuously developing pool (fabric?) of knowledge in academia and beyond.  
\par
The papers in this issue investigate in various ways how we as researchers can sensitize ourselves to difference, how we can fold layers of the world and lived experience into a text, and how we embrace and show nuances. Lindvang and colleagues, writing from the perspective of music therapy, show that the notion of resonance is not only a metaphor used to describe a certain psychological connection between persons, but also a tangible phenomenon consisting of physical vibrations that are experienced both visibly and audibly. This tangible phenomenon is ordered rather than random. They argue that resonance is the basis of human subjectivity and the possibility of synchronization, and point to further metaphors from music, such as pulse, tempo, pitch and harmony. Drawing on Hartmund Rosa’s work, resonance is described as the opposite of alienation, i.e. resonance is fundamentally about connection. Yet at the same time, resonance also requires separateness. It is thus about connection with a person who is \textit{other} than I. Consequently, the article dabbles with some of the ethical implications of this understanding of the notion of resonance. The article develops a nine-step procedure of resonant collaborative writing, creating a bridge between the fields of music and of writing practices.
\par
In both Aagaard’s and Revsbæk’s articles, the focus is turned to the notion of resonance as helpful regarding methodological concerns. Leveraging the work of Mead, Revsbæk explores research analysis as an ‘emergent event’ and develops the concept of ‘resonant experience’ as an analytical tool. Drawing on her own empirical work, Revsbæk shows how researchers’ established conceptions are at times challenged. This calls for the researcher to reorder and re-experience her material and abductively draw on her evoked experiences of resonance. As such, resonance, to Revsbæk, becomes a matter of considering and registering the other in oneself, which allows for new categories to emerge in analysis.  Aagaard draws our attention to resonance as a tool of validity. By situating his discussion in Ricoeur’s distinction between a hermeneutics of suspicion and a hermeneutics of faith, he explores two possible answers to the question of the criteria by which we as researchers should consider a truth valid. He describes both a critical approach which seeks to uncover hidden truths and a phenomenological approach which reveals truths that have yet gone unnoticed, and he shows how these approaches share family resemblances but also differ. Through the article, Aagaard advances the concept of resonance as the reverberance that is struck and the \textit{familiarity} that is evoked. In phenomenology, the researcher seeks to describe the everyday world, thus striving to stir an experience of resonance in the readers of the research. Aagaard suggests that resonance as a methodological tool may help us notice new differences and make the world more articulate. 
\par
Through these contributions, resonance can be considered one answer to the many different versions of the normative question of what ‘good’, high-quality research is. This question is for us closely connected to the ambition of making an impact beyond the easily quantifiable kinds. The world is hurried, and so is academia. This doesn’t mean that we should speed up to keep up, nor does it mean that we should slow down or go into hibernation to restore. As Rosa (2014) says, resonance is not a new slow movement. Resonance is not a retrogressive or nostalgic response to acceleration. Resonance is a connection or reconnection with the world, with our senses and with our foundation enabling us to make informed choices, innovate smartly and, not least, let ourselves be guided not by output indicators, but by doing good and by doing good research.